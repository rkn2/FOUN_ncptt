% v1.04 - May 2023

\documentclass[]{interact}

\usepackage{epstopdf}% To incorporate .eps illustrations using PDFLaTeX, etc.
\usepackage{color} % doh

\usepackage{subfigure}% Support for small, `sub' figures and tables
%\usepackage[nolists,tablesfirst]{endfloat}% To `separate' figures and tables from text if required
\usepackage{longtable}
\usepackage{tabularx}
\usepackage{ltablex}  % This combines longtable and tabularx
\usepackage{colortbl}

\usepackage{gensymb}
\usepackage{multirow}

\usepackage{xcolor}

\usepackage{longtable}
\usepackage{tabularx}
\usepackage{booktabs}

\usepackage[hyphens]{url} % to break urls in bibliography
\usepackage[breaklinks=true]{hyperref} % LaTeX cross references become hyperlinks in pdf output


\usepackage{tikz} % Useful for drawing images, used for creating the frontpage
\usetikzlibrary{positioning} % Additional library for relative positioning 
\usetikzlibrary{calc} % Additional library for calculating within tikz

% Defines a command used by tikz to calculate some coordinates for the front-page
\makeatletter
\newcommand{\gettikzxy}[3]{%
  \tikz@scan@one@point\pgfutil@firstofone#1\relax
  \edef#2{\the\pgf@x}%
  \edef#3{\the\pgf@y}%
}
\makeatother

\usepackage{pdflscape}

\usepackage[square,sort&compress]{natbib}% Citation support using natbib.sty
\bibliographystyle{tfcad.bst} % Use the style of your choice   

\hypersetup{colorlinks=true,linkcolor=blue,urlcolor=blue}
\usepackage{soul}


\usepackage{longtable}
\usepackage{caption}
% what they recommend
\usepackage{array}
\usepackage{enumitem}
\usepackage{amssymb}
\usepackage[demo]{graphicx}
%\widowpenalty10000
%\clubpenalty10000 % to avoid sinle lines at the beginning or bottom of pages

\begin{document}

\articletype{Research Article}

\title{A Data-Driven Approach to Structural Vulnerability Assessment and Intervention Planning for Adobe Structures: A Case Study of Fort Union National Monument}

\author{
\name{Mina Savoroliya \textsuperscript{a}, Evan Oskierko-Jeznacki \textsuperscript{b}, Frank Matero \textsuperscript{c}, Thomas Boothby \textsuperscript{a}, Rebecca Napolitano \textsuperscript{a}, \thanks{CONTACT Rebecca Napolitano. Email: nap@psu.edu}}
\affil{\textsuperscript{a}The Pennsylvania State University, Architectural Engineering Department, ECoRE Building, PA, USA}
\affil{\textsuperscript{b}National Park Service (NPS), Vanishing Treasures Program}
\affil{\textsuperscript{c}University of Pennsylvania (UPenn), Stuart Weitzman School of Design, Historic Preservation Department, Meyerson Hall}
}
\maketitle

\begin{abstract}

We developed a data-driven assessment framework for structural vulnerability in historic adobe at Fort Union National Monument, integrating correlation analysis, Factor Analysis, and regularized regression (Elastic Net) to identify and rank damage conditions affecting wall performance.
On-site surveys of 67 wall sections, archival research, and LiDAR scanning provided the empirical basis for the analysis. Factor Analysis isolated three latent structures of degradation (Sill Deterioration, Surface/Lintel Interaction, and Structural Instability) explaining 52.5\% of total variance.
Model benchmarking across seven regression families (repeated K-fold cross-validation with Wilcoxon statistical testing) confirmed that Ridge and Elastic Net significantly outperformed ensemble methods (Random Forest, Gradient Boosting), with mean cross-validated $R^2$ of 0.76 vs. 0.59 ($p<0.001$).
Wall Height Loss (standardized coefficient 43.0) and Cap Deterioration (54.5) emerged as the dominant predictors, demonstrating that loss of original fabric and water protection are strongly associated with vulnerability. A supplementary validation on a subset of walls ($N=38$) with detailed geometric profiles found no significant correlation between physical slenderness and degradation ($p=0.92$), confirming that condition metrics are the primary indicators of structural health.
Statistically-derived priorities were synthesized with Secretary of Interior preservation standards into intervention matrices, providing field managers with a structured decision framework.
The methodology is constrained by sample size (n=67) and single-site scope but offers a replicable template for integrating geometric and condition data in adobe risk assessment.

\end{abstract}

\begin{keywords}
adobe structures; structural vulnerability assessment; data-driven analysis; intervention planning; structural health monitoring; factor analysis
\end{keywords}

\section{Introduction}

Adobe structures, emblematic of the American Southwest's cultural heritage, represent a unique intersection of history and engineering \cite[]{houben2004}. 
Their earthen composition ties them intrinsically to their landscapes, yet also renders them vulnerable to environmental challenges \cite[]{richards2019, reyes2019, fuentes2019}.
The study of earth-based construction materials has seen significant advancements over the past two decades. 
Early research laid the groundwork for understanding these materials' structural properties. 
\cite[]{J2001} examined the reinforcement behavior of steel bars in cement-stabilized rammed earth, providing insights into reinforcement techniques. 
Building on this, \cite[]{Vasilios2008} investigated the compressive behavior of rammed earth structural elements, offering insights into load eccentricity and slenderness effects.
As the field progressed, durability became a key focus. 
\cite[]{Bui2009} investigated the durability of rammed earth walls, particularly when reinforced with natural hydraulic lime, demonstrating the material's ability to withstand natural weathering. 
This research was complemented by \cite[]{Fratini2011}, who characterized earth-based construction materials in historical structures, identifying optimal clay content for best mechanical performance.

The early 2010s saw a shift towards more sophisticated analysis methods. 
\cite[]{Ciancio2013} proposed a new method for analyzing the structural capacity of unreinforced rammed earth walls under lateral wind forces; \cite[]{Bui2014} focused on failure mechanisms of rammed earth walls, identifying stress concentrations at loaded zones as important considerations; \cite[]{Illampas2014} integrated laboratory testing with finite element simulations for more accurate structural evaluations under horizontal loading conditions, highlighting the influence of weak bonding between masonry units and mortar joints on seismic performance.
Long-term behavior and advanced modeling techniques became prominent in subsequent years. \cite[]{Bui2015} explored the aging and creep behavior of rammed earth, contributing to our knowledge of its long-term structural integrity. 
\cite[]{Bui2016} further advanced numerical analysis techniques by utilizing the discrete element method to simulate rammed earth behavior under various loading conditions.

Recent studies have investigated more specific aspects of earth-based materials. 
\cite[]{Characterization-mechanical} offered an overview of unstabilized rammed earth, examining aspects like compressive strength and thermal performance; \cite[]{non-industrial} examined the compression behavior of rammed earth, establishing a basis for understanding its mechanical response under varying loading conditions.
Environmental factors have also been a recent focus. \cite[]{Effectofmoisture} found that slight increases in moisture content did not significantly affect wall strength, while \cite[]{Impactofrelative} explored the effects of relative humidity changes on compacted earth's mechanical properties. 
Lastly, \cite[]{Ioannou} provided a stress-strain equation for adobe bricks, essential for structural design, further refining our understanding of these materials.

Seismic behavior has been a focus for several researchers. 
\cite[]{Varum2015} conducted in-depth studies on the mechanical properties of adobe units and mortars, performing strength and joint shear tests. 
\cite[]{Sathiparan2018} examined the importance of roof and diaphragm connectivity in seismic performance, revealing the benefits of high-tensile polypropylene strap retrofitting. \cite[]{Xekalakis2023} demonstrated how wooden ring beams significantly contribute to seismic resilience in adobe structures. 
\cite[]{Rafi2022} proposed a seismic strengthening scheme using steel wire mesh and cement-sand mortar, validated through shaking table tests.
\cite[]{hart2023} experimentally subjected extant adobe block construction walls to a local 100-year return interval rainfall intensity, highlighting the importance of data-driven modeling in targeting preservation methods effectively.

Data-driven approaches, particularly Principal Component Analysis (PCA) and correlation analysis, have gained prominence in structural health monitoring, offering new avenues for preservation and risk mitigation in historic adobe structures. 
The evolution of these techniques has significantly enhanced our ability to detect and analyze structural changes while accounting for environmental factors.
Early research in this field focused on developing methods to differentiate between environmental effects and actual structural damage. 
\cite[]{Yan2005} introduced a PCA-based method for detecting structural damage that integrated environmental factors without direct measurement. 
They further refined this concept in a subsequent study \cite[]{Yan2005b}, applying local PCA in a clustered data environment to address non-linear complexities. 
This work laid the foundation for more sophisticated analysis techniques in structural health monitoring.
Building on these advancements, \cite[]{Bellino2010} demonstrated PCA's effectiveness in both Linear Time-Invariant and Linear Time-Varying systems. 
Through experiments on a clamped-free beam and numerical simulations of a train crossing a railway bridge, their work showcased PCA's ability to discern damage from environmental influences, further validating its utility in structural health monitoring.

The application of these techniques to real-world structures was exemplified by \cite[]{Magalhães}, who analyzed data from a concrete arch bridge in Porto. 
They employed static and dynamic regression models enhanced by PCA to establish relationships between natural frequencies and influencing factors, demonstrating the practical application of these methods in complex, real-world environments.
For vibration-based structural health monitoring, \cite[]{Ubertini2013} implemented a system for the San Pietro belltower in Perugia. 
Using ambient vibration tests and machine learning methods including PCA, they were able to identify post-earthquake anomalies in structural behavior. Their approach, which employed novelty analysis for damage detection, showcased the potential of these techniques in preserving historic structures.
More recently, \cite[]{Zucconi2017} expanded the application of PCA to large-scale scenario analysis. 
They developed a model to predict the habitability of unreinforced masonry buildings post-earthquake, utilizing PCA to analyze over 60,000 buildings and identify seven parameters impacting usability. 
This study demonstrated the potential of machine learning techniques for scenario analysis and planning in regions with similar building characteristics, highlighting the scalability of these methods.

As technology continues to advance, these methods are likely to play an increasingly important role in safeguarding not only civil infrastructure but also our architectural heritage.
Within preservation practice, \cite[]{Dehghan} used PCA to assess the effectiveness of historic buildings repurposed as boutique hotels. 
Their study distilled 24 indicators into three principal components, providing actionable insights for stakeholders in making user-centric improvements.
\cite[]{kallas2024understanding} utilized Elastic Net and other ML models to analyze structural vulnerability in unreinforced masonry buildings following the 2020 Beirut port explosion, identifying building geometry and age as key features shaping damage severity. 
\cite[]{kallas2025kakhovka} applied similar techniques to assess masonry damage after the 2023 Kakhovka Dam breach in Ukraine, demonstrating that contextual factors such as flooding duration and foundation type govern damage patterns. 
The present study demonstrates the power of data-driven techniques in enhancing our understanding and management of historic adobe structures, offering new tools for preservation and risk mitigation.

\section{Research aim}

We develop a data-driven approach for assessing structural vulnerability in adobe structures at Fort Union National Monument (FOUN), integrating correlation analysis, Factor Analysis, and Regularized Regression (Elastic Net) to identify the dominant drivers of vulnerability and optimize assessment protocols. By recovering the implicit weights of the scoring system, the analysis validates which conditions (e.g., moisture ingress vs. mechanical cracking) most strongly govern the overall vulnerability rating, enabling more efficient resource allocation. The analysis quantifies relationships among degradation mechanisms and translates statistical patterns into prioritized intervention strategies that satisfy both preservation ethics and structural requirements. 
The approach bridges traditional condition assessment with machine learning, providing empirical evidence for resource allocation decisions in contexts where degradation mechanisms interact through multiple pathways.


\section{Case study}

\begin{figure}[h]
    \centering
\includegraphics[width=1\textwidth]{Images/FOUN.png}
    \caption{Fort Union National Monument \cite[]{map}}
    \label{fig:whole_site}
\end{figure}

FOUN is a historically significant site adjacent to the historic Santa Fe Trail. It was established in 1851, shortly after New Mexico became a U.S. territory. Over its history, three distinct structures were built in proximity, each serving a different strategic purpose. This study focuses on the hospital building located at the southeast end of the site, which is part of the third instance of Fort Union (Figure~\ref{fig:whole_site}). 
\begin{figure}[h]
    \centering
\includegraphics[width=0.8\textwidth]{Images/land.png}
    \caption{FOUN site plan \cite[]{map}}
    \label{fig:whole_site}
\end{figure}
The Third Fort hospital at FOUN was constructed between late 1863 and early 1864 and remained in use until 1891 when the last soldiers left for Fort Wingate and while Fort Union was vacated. 
Since its abandonment in 1891, the hospital complex (Figure~\ref{fig:fort_union_hospital_plan}), along with the rest of the fort, has been subject to various environmental and human-induced challenges. 
These include exposure to the elements, natural deterioration processes, and periods of limited maintenance before its designation as a National Monument. 
The site also experienced some salvage of materials prior to NPS management.

\begin{figure}[h!]
    \centering
    \includegraphics[width=0.8\textwidth]{Images/hospital.png}
    \caption{The plan of hospital at FOUN}
    \label{fig:fort_union_hospital_plan}
\end{figure}

Since the 1950s the National Park Service (NPS) has made continual efforts to stabilize the remnant walls, including an array of wall bracings and other supports, but certain problems with the walls require unusual efforts to maintain them in place and preserve the remaining original fabric. The hospital complex includes the hospital itself, the inner courtyard, and areas enclosed by the surrounding compound wall foundations and exterior to the hospital and inner courtyard. 

\section{Materials and methods}
\subsection{Data collection}
\subsubsection{For adobe walls}
The dataset documents adobe structural damage at FOUN. Each entry represents a wall section identified by a unique ID and includes features and scores reflecting the condition of the adobe structures. 
Data sources include photographic documentation, LiDAR scans, architectural plans provided by the National Park Service, and archival research identifying past treatment efforts. 
Data was collected using a rapid assessment survey (RAS), complemented by an illustrated glossary to standardize damage severity classifications.
Figure~\ref{fig:RASsample} shows a sample of this illustrated glossary, and additional information can be found in Section \ref{sec:ras}.

\begin{figure}[h!]
    \centering
    \includegraphics[width=1\textwidth]{Images/RASsample.png}
    \caption{Example pages from the illustrated glossary that accompanied the RAS}
    \label{fig:RASsample}
\end{figure}

The dataset includes the following features, each with an associated scoring system to quantify the level of damage or presence of specific attributes. 
Specifics about orientation of damages can be found in Section \ref{sec:orient}. 
\begin{itemize}
    \item \textbf{Coat Cracking:}  Represents the level of cracking in the shelter coat.
        \begin{itemize}
            \item 0: No Damage
            \item 1-5: Damage Level (increasing severity)
        \end{itemize}
    \item \textbf{Coat Loss:} Represents the level of loss in the shelter coat.
        \begin{itemize}
            \item 0: No Damage
            \item 1-5: Damage Level (increasing severity)
        \end{itemize}
    \item \textbf{Structural Cracking:} Represents the presence and severity of structural cracks.
        \begin{itemize}
            \item 0: No Damage
            \item 1-5: Damage Level (increasing severity)
            \item 6: Wall Destroyed
        \end{itemize}

    \item \textbf{Cracking at Wall Junction:} Represents the presence and severity of cracks at wall junctions.
        \begin{itemize}
            \item 0: No Damage
            \item 1-5: Damage Level (increasing severity)
            \item 6: No Wall Junction
        \end{itemize}
    \item \textbf{Lintel Deterioration:} Represents the condition of the lintel (if present).
        \begin{itemize}
            \item 0: No Damage
            \item 1-5: Damage Level (increasing severity)
            \item 6: No Lintel
        \end{itemize}
    \item \textbf{Foundation Displacement:} Represents the displacement of the foundation.
        \begin{itemize}
            \item 0: No Damage
            \item 1-5: Damage Level (increasing severity)
            \item -1: Missing Data
        \end{itemize}
    \item \textbf{Foundation Mortar Condition:} Represents the condition of the mortar in the foundation.
        \begin{itemize}
            \item 0: No Damage
            \item 1-5: Damage Level (increasing severity)
            \item -1: Missing Data
        \end{itemize}
    \item \textbf{Sill:} Represents the condition of the sill (if present).
        \begin{itemize}
            \item 0: No Sill
            \item 1-5: Damage Level (increasing severity)
            \item 6: Sill Destroyed
        \end{itemize}
    \item \textbf{Cap Deterioration:} Represents the condition of the wall cap (if present).
        \begin{itemize}
            \item 0: No Cap
            \item 1-5: Damage Level (increasing severity)
            \item 6: Wall Destroyed
        \end{itemize}
    \item \textbf{Surface Loss at Top, Mid, and Low Level:} Represents the level of surface erosion at different wall levels.
        \begin{itemize}
            \item 0: No Damage
            \item 1-5: Damage Level (increasing severity)
        \end{itemize}
    \item \textbf{Out of Plane:} Represents the degree to which the wall is leaning or racked.
        \begin{itemize}
            \item 0: No Damage
            \item 1-5: Damage Level (increasing severity)
        \end{itemize}
    \item \textbf{Height:} Represents the remaining height of the wall.
        \begin{itemize}
            \item 1: No Damage (Full Height)
            \item 2-5: Damage Level (decreasing height)
        \end{itemize}
    \item \textbf{Foundation Height:}  Represents the exposed height of the foundation in inches. A blank cell indicates the foundation is not exposed.
    \item \textbf{Animal Activity:} Represents the level of animal activity.
        \begin{itemize}
            \item 0: No Damage
            \item 1-5: Damage Level (increasing severity)
        \end{itemize}
    \item \textbf{Foundation Stone Deterioration:} Represents the condition of the foundation stones.
        \begin{itemize}
            \item 0: No Damage
            \item 1-5: Damage Level (increasing severity)
            \item -1: Missing Data
        \end{itemize}
    \item \textbf{Bracing Score:} Represents the condition of any bracing present.  Calculated based on observed damage to the bracing.
        \begin{itemize}
            \item 0: No Bracing
            \item 1-5: Damage Level (decreasing condition)
        \end{itemize}
    \item \textbf{Fireplace:} Indicates the presence and proximity of a fireplace.
        \begin{itemize}
            \item 0: No Fireplace
            \item 1: Has Fireplace
            \item 2: Adjacent Fireplace
        \end{itemize}
    \item \textbf{Treatment History:} A binary indicator aggregating 41 specific historical intervention records (e.g., 1960s capping, recent repointing). Due to the sparsity of individual treatment types, these were combined into a single feature indicating the presence of any past intervention. (Note: In the final analysis, this variable was excluded as it exhibited zero variance across the sample).
        \begin{itemize}
            \item 0: Not Treated
            \item 1: Treated
        \end{itemize}
    \item \textbf{Bracing:} Indicates the presence of bracing.
        \begin{itemize}
            \item 0: No Bracing
            \item 1: Has Bracing
        \end{itemize}

\end{itemize}

\subsection{Analysis methods}
The Pearson correlation was initially employed to explore the linear interrelationships among the various parameters influencing adobe structure behavior. 
By calculating Pearson correlation coefficients for each pair of parameters, we quantified the strength and direction of linear relationships. 
Coefficients approaching +1 or -1 indicated strong positive or negative linear associations, respectively, while values near 0 suggested a negligible linear relationship. 
This process yielded a correlation matrix quantifying the linear interdependencies between all parameter pairs. P-values for each correlation coefficient were calculated to assess statistical significance. 
A small p-value (e.g., $p < 0.05$) indicated strong evidence against the null hypothesis (no correlation), suggesting that the observed correlation was statistically significant and unlikely due to random chance. 
We then focused on interpreting the statistically significant correlations, moving beyond simply identifying the `what' to understanding the `why' behind these relationships. 
This involved analyzing the context of each correlation to discern the underlying theoretical or practical reasons driving the observed association, thereby translating statistical findings into actionable insights for decision-making.

Factor Analysis was employed to identify latent variables that explain the pattern of correlations within the observed degradation metrics. 
Unlike PCA, which focuses on explaining the total variance, Factor Analysis focuses on the common variance among variables, making it particularly suitable for identifying underlying constructs such as ``Structural Integrity" or ``Surface Wear" from a set of observed damage indicators. 
We used the FactorAnalyzer module \hl{Becca cite} in Python with varimax rotation to achieve a simpler and more interpretable structure. 
This method allowed us to group multiple degradation scores into a smaller number of meaningful factors, reducing the dimensionality of the data while preserving the essential information about the structural health.

Building upon the insights gained from both correlation analysis and Factor Analysis, we further refined our understanding by employing Regularized Linear Regression (Elastic Net). 
Elastic Net addresses this by combining L1 (LASSO) and L2 (Ridge) regularization penalties, effectively handling multicollinearity and performing feature selection by shrinking irrelevant coefficients to zero \cite[]{zou2005regularization}.
The model predicts the overall degradation score (Total Scr) based on geometric features (e.g., wall height, foundation height) and treatment history. 
The $\text{Total Scr}$ serves as the target variable for the regression, consolidating various degradation scores into a single, continuous metric of overall wall health. 
This composite score is a weighted sum of normalized component scores reflecting three primary damage categories (details on sub-scores are provided in the Appendix, Table \ref{tab:metrics}): global wall instability, surface coat integrity, and localized elevation damage. It is calculated as:
$$
\text{Total Scr} = \text{Wall Weight Score} + \sum_{i=1}^{4} \text{Coat } i \text{ Weight Score} + \sum_{j=1}^{2} \text{Elevation } j \text{ Weight Score}
$$
The $\text{Wall Weight Score}$ aggregates severe structural features (Cap Deterioration, Structural Cracking, Out of Plane, Lintel Deterioration, etc.), while the four $\text{Coat Weight Scores}$ quantify shelter coat degradation. The $\text{Elevation Weight Score}$ focuses on damage localized at the base and sill. 


To ensure the generalizability of our findings across algorithmic approaches, we benchmarked seven regression model families: Linear Regression, Ridge Regression (L2-regularized), Elastic Net (L1+L2 regularized), Decision Tree, Random Forest, Support Vector Regression (SVR), and Gradient Boosting (XGBoost equivalent), alongside a Naive Baseline (predicting the mean) to establish a performance floor.
Repeated K-fold cross-validation was implemented with five folds and five repeats, yielding 25 evaluation rounds per model.
This design provides 25 independent performance estimates per model, characterizing stability and distribution.

The model achieving the best mean $R^2$ score was identified, and paired Wilcoxon signed-rank tests were performed on the 25 fold-wise $R^2$ differences to compare competing models.
The Wilcoxon test is appropriate for paired, non-normal data and is recommended for classifier comparisons \cite[]{demvsar2006statistical}.
Holm-Bonferroni correction for multiple comparisons was applied to ensure statistical rigor.
Models with corrected p-values exceeding 0.05 were deemed statistically indistinguishable from the best performer.

The benchmark revealed that regularized linear models (Ridge and Elastic Net) significantly outperformed complex non-linear ensembles (Random Forest, Gradient Boosting).
Ridge Regression achieved the highest mean $R^2$ (0.76), far exceeding the Naive Baseline ($R^2 = -0.10$), confirming the model captures significant signal beyond random chance. Elastic Net ($R^2 = 0.75$) was statistically indistinguishable from Ridge ($p_{adj} = 0.17$).
Given their statistical equivalence, we selected Elastic Net as the primary model for interpretation due to its ability to perform feature selection via L1 regularization, offering a parsimonious explanation of the degradation drivers.
All predictors were standardized to z-scores (mean=0, SD=1) prior to modeling, allowing regression coefficients to serve as direct measures of feature importance. Missing values in foundation variables were handled using mean imputation. The Elastic Net model was configured with a mixing parameter ($\alpha$) of 1.0 and an L1 ratio of 0.5, balancing LASSO and Ridge penalties. Residual diagnostics for the Elastic Net model confirmed homoscedasticity and approximate normality, with no highly influential outliers detected (see Appendix D, Figure \ref{fig:residual_diagnostics}). Variance Inflation Factors (VIF) for the active predictors ranged from 1.4 to 8.2 (median 1.9), confirming moderate multicollinearity (particularly between sill measurements) that justified the use of regularized regression.

\begin{table}[h]
\centering
\caption{Model Benchmarking Results (Repeated K-Fold Cross-Validation, 5 folds $\times$ 5 repeats)}
\label{tab:model_benchmark}
\begin{tabular}{lccc}
\hline
Model & Mean $R^2$ & Std Dev & Status vs Best \\
\hline
Ridge & 0.764 & $\pm$0.132 & Best \\
Elastic Net & 0.750 & $\pm$0.082 & $p=0.17$ \\
Linear Regression & 0.744 & $\pm$0.144 & $p<0.001$ \\
Gradient Boosting & 0.610 & $\pm$0.150 & $p<0.001$ \\
\hline
\multicolumn{4}{l}{\footnotesize Wilcoxon signed-rank test with Holm-Bonferroni correction} \\
\multicolumn{4}{l}{\footnotesize Additional models tested: Random Forest, Decision Tree, SVR} \\
\multicolumn{4}{l}{\footnotesize Full results for all 7 models available in Appendix \ref{appendix:benchmarking}} \\
\end{tabular}
\end{table}

\subsection{Data synthesis methods}
Intervention matrices graphically represent the logic of intervention decisions, reminding stakeholders to consider multiple perspectives and options\cite[]{harris2001building}. The tool avoids formulaic responses to commonly encountered deterioration mechanisms by facilitating a structured evaluation of the various options, weighing their advantages and disadvantages, and considering cost, risk, efficacy, and potential outcomes. It also illustrates the complexity of decision-making in these contexts \cite[]{harris2001building}. In these matrices, the horizontal axis represents the NSC, while the vertical axis represents various intervention approaches: abstention, mitigation, reconstitution, substitution, circumvention, and acceleration as defined by \cite[]{harris2001building}. The cells within the matrix represent a qualitative assessment of the effectiveness, cost, and potential risks associated with each intervention approach for a given NSC. For example, a cell might indicate that repointing mortar joints is highly effective for addressing cracking at wall junctions but has a moderate cost and a low risk of negatively impacting the historical fabric.
The initial step in developing an intervention matrix involves determining the NSCs.
This process uses the outputs of ranked Pearson correlations, Factor Analysis, and Elastic Net to identify which features show the strongest associations with others.  

\section{Results and discussion}

\subsection{Understanding damages to adobe walls}

\subsubsection{Correlation Analysis for Damage to Adobe Walls}
Pearson correlation analysis was performed to quantify the linear relationships between degradation metrics and structural features across the 67 wall sections (two-tailed tests). 
All p-values were adjusted using the Bonferroni correction to account for multiple comparisons (corrected $\alpha = 0.001$ for 55 pairwise comparisons). 
The analysis revealed several strong associations with the Total Score (Total Scr), which serves as an aggregate measure of degradation severity. 
Cap Deterioration exhibited the strongest correlation with Total Scr ($r = 0.61$, $p < 0.001$), followed by Out of Plane movement ($r = 0.58$, $p < 0.001$), Height ($r = 0.55, p < 0.001$), and Structural Cracking ($r = 0.53, p < 0.001$). 
These positive associations indicate that taller walls and those exhibiting more pronounced structural issues such as leaning or cap failure tend to accumulate higher overall degradation scores. 
\hl{Becca add pics of this!}

The correlation between Bracing and Total Scr ($r = 0.26$, $p = 0.03$) warrants particular attention. 
This positive association may reflect reverse causality: bracing is typically installed as a reactive measure on walls that have already experienced significant deterioration rather than as a preventive intervention. 
\hl{Becca add pics of this!}
Caution is warranted in interpreting this correlation, as our cross-sectional data cannot establish the temporal sequence of bracing installation relative to damage accumulation. 
Future work with historical records of bracing installation dates could disentangle this relationship through temporal analysis or propensity score matching.

\begin{figure}[h]
    \centering
    \includegraphics[width=0.8\textwidth]{Images/correlation_heatmap.png}
    \caption{Correlation Matrix of Key Degradation Metrics.}
    \label{fig:correlation_heatmap}
\end{figure}

\subsubsection{Factor Analysis of Degradation Patterns}

Factor Analysis was employed to identify latent variables underlying the observed degradation patterns. 
This technique differs from simple correlation in that it seeks to explain the common variance among multiple observed variables through a smaller number of unobserved factors. 
We proceeded with Factor Analysis as Bartlett's test strongly rejects variable independence ($\chi^2 = 173.14$, $df = 28$, $p < 0.001$). While the sample size ($N=67$) is modest, the resulting factor solution is validated post-hoc by the strength and interpretability of the loadings. Key variables exhibit loadings exceeding 0.70 (e.g., Sill 1: 0.94, Lintel Deterioration: 0.82), confirming that the extracted factors represent robust, physically distinct failure modes rather than statistical artifacts.
Results should be interpreted as exploratory structure identification rather than definitive factor solutions.
Bootstrap validation of factor stability was not conducted given sample size constraints; replication at larger adobe sites would strengthen confidence in the factor structure.

A three-factor model with varimax rotation was selected based on eigenvalue-greater-than-one rule (Kaiser criterion) \cite[]{kaiser1960application}, scree plot inflection point, and interpretability of factor structure. 
The three extracted factors explained 52.5\% of the total variance, which is considered acceptable for exploratory studies in complex field conditions, with individual factor contributions of 23.5\% (Factor 1), 16.5\% (Factor 2), and 12.4\% (Factor 3). 
Varimax rotation was chosen to achieve orthogonal (uncorrelated) factors for simpler interpretation. 
Supplementary analysis using oblique rotation (promax) yielded similar factor structure (mean max loading varimax: 0.758, promax: 0.765), supporting the choice of varimax for final interpretation. 
Communalities (proportion of variance in each variable explained by the extracted factors) are reported in Table \ref{tab:communalities}, with all values exceeding 0.4, indicating adequate representation of each variable by the factor solution. 
The factor loadings are presented in Table \ref{tab:fa_loadings}.

\begin{table}[h]
\centering
\caption{Factor Analysis Loadings (Varimax Rotation)}
\label{tab:fa_loadings}
\begin{tabular}{lccc}
\toprule
\textbf{Variable} & \textbf{Factor 1} & \textbf{Factor 2} & \textbf{Factor 3} \\
\midrule
Sill 1 & \textbf{0.94} & - & - \\
Sill 2 & \textbf{0.97} & - & - \\
Coat 1 Cracking & - & 0.43 & \textbf{0.60} \\
Coat 1 Loss & - & - & \textbf{0.51} \\
Lintel Deterioration & - & \textbf{0.82} & - \\
Coat 2 Cracking & - & \textbf{-0.54} & - \\
Structural Cracking & - & -0.33 & 0.36 \\
Out of Plane & - & - & \textbf{0.48} \\
\bottomrule
\end{tabular}
\end{table}

\begin{table}[h]
\centering
\caption{Communalities for Factor Analysis Variables}
\label{tab:communalities}
\begin{tabular}{lc}
\toprule
\textbf{Variable} & \textbf{Communality} \\
\midrule
Sill 1 & 0.90 \\
Sill 2 & 0.94 \\
Coat 1 Cracking & 0.55 \\
Coat 1 Loss & 0.29 \\
Lintel Deterioration & 0.69 \\
Coat 2 Cracking & 0.30 \\
Structural Cracking & 0.25 \\
Out of Plane & 0.27 \\
\bottomrule
\end{tabular}
\end{table}

Communalities (proportion of variance explained by the 3-factor solution) are reported in Table \ref{tab:communalities}. While some values are low (e.g., Structural Cracking 0.25), the Sills show very high communality (>0.90), indicating the factor solution captures these features well.
The factor loadings are presented in Table \ref{tab:fa_loadings}.

The first factor is heavily loaded on Sill 1 (0.94) and Sill 2 (0.97). This factor may primarily reflect measurement redundancy (sill condition assessed on opposite orientations of the same wall) rather than a distinct latent construct.\footnote{Factor 1 loads almost exclusively on two highly correlated sill measurements (r=0.99), likely reflecting measurement redundancy rather than a latent construct. We retain this factor to preserve the analytical framework, but acknowledge its interpretation is limited. Sensitivity analysis collapsing Sill 1 and Sill 2 into a single variable yielded substantively similar results for Factors 2-3 (see Appendix C).} However, we retain this factor because: (1) the high communalities (Sill 1: 0.99, Sill 2: 1.00) indicate these variables explain substantial shared variance, (2) sill deterioration represents a specific, localized failure mode requiring targeted interventions distinct from general wall degradation, and (3) while Sill 1 and Sill 2 are highly correlated (r = 0.99, p < 0.001), they are not perfectly identical (loading difference of 0.03), suggesting some orientation-specific exposure effects. 
The factor name ``Sill Deterioration" reflects that sills function as water-shedding elements at the base of openings; their deterioration, while often localized, can facilitate water ingress into the wall core.

The second factor, named ``Surface and Lintel Interaction", exhibits high loadings for Coat 1 Cracking (0.72) and Lintel Deterioration (0.61). 
This factor structure suggests an association between these two forms of damage. 
Surface coat failure around openings may be associated with wooden lintel moisture exposure, or potentially reflects stresses in surrounding plaster associated with lintel deflection. 
This finding has implications for intervention planning, suggesting that lintels and their surrounding surface coats may benefit from coordinated treatment approaches.

The third factor, ``Global Structural Instability", groups Structural Cracking (0.50), Out of Plane movement (0.40), and Coat 2 Cracking (0.58). 
The inclusion of Coat 2 Cracking (Orientation 2) in this structural factor warrants explanation. 
We hypothesize this reflects orientation-specific exposure: the West-facing walls (Orientation 2 and 3) are subjected to the prevailing westerlies and associated severe weather, including wind-driven rain.
If Orientation 2 walls experience greater environmental stress (wind-driven rain, solar exposure), surface coat damage on this face may co-occur with structural instability associated with that same exposure.
Alternatively, if structural movement (out-of-plane, cracking) concentrates on particular wall orientations due to asymmetric loading or foundation settlement patterns, coating on those faces may exhibit increased distress. 
This clustering suggests these damage forms share common underlying associations with structural issues such as foundation settlement, loss of lateral restraint, or material degradation in the wall core. 
Interventions targeting this factor should address underlying structural deficiencies rather than merely treating surface manifestations.

\subsubsection{Feature Importance Analysis Using Elastic Net Regression}

Elastic Net regression identified the most significant predictors of overall degradation by assigning penalized coefficients to each feature. 
The results are presented in Figure \ref{fig:feature_importance}.

\begin{figure}[h]
    \centering
    \includegraphics[width=0.8\textwidth]{Images/feature_importance.png}
    \caption{Standardized Regression Coefficients for Predicting Total Degradation Score (Elastic Net).}
    \label{fig:feature_importance}
\end{figure}

The top predictors of degradation were Cap Deterioration (Coefficient = 54.5), Out of Plane movement (43.3), and Height Loss (43.0).
Structural Cracking (30.8) and Cracking at Wall Junction (27.7) also showed significant positive associations.
These results align with the benchmark findings, where linear models consistently outperformed non-linear ones ($R^2_{Ridge}=0.76$ vs $R^2_{RF}=0.59$, $p < 0.001$).
This statistical superiority of linear models confirms that degradation scoring follows a predictable, additive logic where specific defects cumulatively drive the overall condition state.

\subsubsection{Predictive Power of Geometric Factors}

To isolate the predictive power of geometric and contextual features without damage indicator confounding, we conducted a supplemental Elastic Net analysis predicting Total Score using only non-damage variables (excluding Height Loss). This geometric-only model achieved a cross-validation $R^2 \approx 0.03$, indicating that available geometric and contextual metadata alone are insufficient to predict degradation state.
\subsubsection{Geometric Validation Sub-study}
To test the hypothesis that geometric slenderness drives degradation, a supplementary analysis was conducted on a subset of 38 walls for which detailed profilometry data was available. The Slenderness Ratio (Height/Width) was calculated from the physical profiles and correlated with the Total Degradation Score. A simplified geometric slenderness ratio (Height/Width) showed no correlation with degradation ($r = -0.016$, $p = 0.925$). However, this metric does not account for boundary conditions, material properties, or load eccentricity that govern true structural slenderness per Euler buckling theory. The null result suggests simple dimensional ratios are insufficient proxies for stability risk; future work should incorporate effective height calculations and finite element analysis to assess elastic instability.

To physically contextualize this statistical finding, a simplified structural stability check was performed on the geometric subset (see Appendix E for details). The analysis demonstrated that while no walls exceeded the critical buckling load (mean Factor of Safety $> 40$), 95\% of walls developed net tensile stresses under moderate wind loading (20 psf) that exceeded the negligible tensile capacity of adobe. This confirms that failure is governed by tensile cracking—a mechanism directly exacerbated by the condition-based factors (e.g., material loss, existing cracks) identified by the Elastic Net model—rather than by elastic instability.




\subsection{Identifying significant conditions from results}

The development of intervention matrices requires identifying damage conditions from the multivariate analyses. 
Based on the Factor Analysis and Elastic Net results, we identified four primary conditions warranting intervention focus:

\textbf{(1) Structural Instability (Factor 3, EN rank 2-3):} This encompasses Structural Cracking  and Out of Plane movement, grouping into Factor 3 alongside Coat 2 Cracking. 
This cluster suggests coupled structural failure requiring intervention addressing underlying causes (foundation issues, loss of lateral restraint) rather than merely treating surface manifestations.

\textbf{(2) Surface Coating Degradation (Factor 3, EN rank 1):} Coat 2 Cracking emerged as the single strongest predictor, loading heavily on Factor 3. 
Its association with structural instability suggests coating failure is strongly linked to structural deterioration, potentially serving as both an indicator and an aggravating factor by exposing adobe to moisture infiltration.

\textbf{(3) Sill Deterioration (Factor 1):} This represents a distinct, localized vulnerability. 
While moderate in predictive importance, its independence as Factor 1 indicates a specific failure mode requiring targeted intervention separate from general wall treatments.

\textbf{(4) Surface and Lintel Interaction (Factor 2):} Coat 1 Cracking and Lintel Deterioration group as Factor 2, suggesting a coupled relationship where coating damage and lintel decay are closely linked, particularly around openings.

Note that we do not establish ``High Geometric Exposure" as an NSC in the intervention sense, as height and foundation exposure are intrinsic wall characteristics that cannot be directly ``treated." 
However, these geometric factors inform prioritization: walls combining high geometric risk with moderate current damage warrant early intervention to mitigate potential progression to severe structural instability.
Table \ref{tab:significant_features} lists the features identified as significant in the analysis, detailing their inclusion or exclusion from the intervention matrix and the rationale.

\begingroup
\footnotesize
\setlength\tabcolsep{4pt}
\begin{longtable}{>{\raggedright\arraybackslash}p{2.5cm} c c c c >{\raggedright\arraybackslash}p{5cm}}
    \caption{Overview of significant features identified in analysis, their status in the intervention matrix, and justifications for exclusion.}
    \label{tab:significant_features} \\ % Add caption here
    \toprule
    \textbf{Key Features} & \textbf{Incl.} & \textbf{Corr.} & \textbf{FA} & \textbf{EN} & \textbf{Reason for Exclusion (if Not Included)} \\
    \midrule
    \endfirsthead

    \multicolumn{6}{c}{\tablename\ \thetable\ -- \textit{Continued from previous page}} \\
    \toprule
    \textbf{Key Features} & \textbf{Incl.} & \textbf{Corr.} & \textbf{FA} & \textbf{EN} & \textbf{Reason for Exclusion (if Not Included)} \\
    \midrule
    \endhead

    \midrule
    \multicolumn{6}{r}{\textit{Continued on next page}} \\
    \endfoot

    \bottomrule
    \endlastfoot

    Sill Deterioration & Yes & & \checkmark & & Distinct Factor 1 \\
    Structural Cracking & Yes & \checkmark & \checkmark & & Part of Factor 3 (Instability) \\
    Out of Plane & Yes & \checkmark & \checkmark & & Part of Factor 3 (Instability) \\
    Coat Cracking & Yes & & \checkmark & & Part of Factor 2 \\
    Lintel Deterioration & Yes & & \checkmark & & Part of Factor 2 \\
    Foundation Height & Yes & & & \checkmark & Top Predictor in RF \\
    Height Loss & No & \checkmark & & \checkmark & Associated with material loss, not a treatable condition \\
    Bracing & Yes & \checkmark & & \checkmark & Significant predictor \\
\end{longtable}
\normalsize
\endgroup

\subsection{FOUN Intervention Matrices}

With the NSCs identified through the combined application of correlation analysis, Factor Analysis, and Elastic Net regression, intervention matrices were constructed following the framework proposed by \cite[]{harris2001building}. The horizontal axis of each matrix represents different intervention approaches: abstention, mitigation, reconstitution, substitution, circumvention, and acceleration. The vertical axis lists the NSCs, ranked according to their importance as determined by the factor loadings and feature importance scores. Each cell in the matrix contains a numerical score representing the product of the NSC priority and the intervention approach's alignment with preservation standards. Higher scores indicate interventions that address high-priority conditions while maintaining fidelity to preservation principles.

Tables \ref{tab:intervention_matrix_adobe} and \ref{tab:intervention_matrix_foundation} present the intervention matrices for adobe wall degradation and foundation degradation, respectively. 
The scoring system incorporates two primary criteria, combined multiplicatively to yield cell scores.

The final NSC Priority Scores (0-10) were determined through a structured process of expert elicitation and consensus rather than direct mathematical calculation. 
The statistical findings, including Factor Analysis loadings and Elastic Net feature importance, were utilized as quantitative evidence to identify and validate the core set of NSCs relevant to degradation.
These quantitative results established the relevant degradation phenomena (e.g., Structural Instability, Sill Deterioration) but were not used for initial score weighting. 
The final 0-10 priority ranking was assigned exclusively by multiple expert raters; this expert judgment allowed for the application of real-world context and structural risk assessment. 
Raters specifically adjusted scores based on the immediacy of risk (prioritizing conditions posing an immediate safety concern), cascading effects (accounting for how one condition accelerates others), and operational feasibility (considering NPS management priorities). 
This methodology ensures that the intervention priorities are grounded in practical reality and risk assessment, while using the quantitative data to define the most statistically relevant degradation phenomena.


% Each intervention approach is assigned a Preservation Standards Score (0-6) reflecting compatibility with the Secretary of the Interior's Standards for the Treatment of Historic Properties \cite[]{weeks1995secretary}.
% Mitigation receives the highest score (6), representing actions that slow deterioration while preserving maximum original fabric (Standard 5: ``preserve distinctive features''; Standard 6: ``repair rather than replace''), such as applying shelter coats or installing drainage improvements.
% Circumvention scores 5, representing interventions addressing underlying causes without altering the historic material (Standard 1: ``minimal alteration''), such as structural bracing or windbreaks.
% Reconstitution scores 4, representing rebuilding deteriorated elements using matching original materials and techniques (Standard 6: ``replacement in kind''), such as rebuilding adobe sills with original mix.
% Substitution scores 2, representing replacement with improved materials that sacrifice authenticity for durability (acceptable only when originals are irreparably damaged, per Standard 6), such as cedar lintels replacing original wood.
% Abstention and Acceleration/Demolition both score zero, as neither addresses the NSC; abstention may be appropriate for low-priority conditions under monitoring protocols.
% Scores within each NSC are normalized such that the preservation-preferred approaches (mitigation, circumvention) receive highest weight for adobe structures requiring active conservation.
Each intervention approach was assigned a Preservation Standards Score (0-6) reflecting its compatibility with the \textit{Secretary of the Interior's Standards for the Treatment of Historic Properties}. 
These scores were not derived from a linear formula but were determined through a consensus process involving multiple expert raters, who evaluated each intervention's alignment with the preservation hierarchy mandated by the Standards. In this instance mitigation (Score 6) received the highest score, representing actions that slow deterioration while preserving the maximum amount of original fabric, aligning with the Standards' requirement to repair rather than replace deteriorated features whenever possible. 
Circumvention (Score 5) represents interventions that address underlying causes of failure without significantly altering the historic material, consistent with guidelines for stabilization as a preliminary measure to protect the property.
Reconstitution (Score 4) denotes the replacement of extensively deteriorated elements in kind, matching the original in material, design, color, and texture, which is the preferred method when repair is no longer feasible.
Ranked lower is Substitution (Score 2), which involves replacement using compatible substitute materials; while permissible under the Guidelines when in-kind replacement is not technically or economically feasible, it sacrifices material authenticity. Finally, Abstention and Acceleration (Score 0) were assigned values of zero, as failing to undertake measures to protect the property violates the core tenets of preservation and potentially leads to the loss of historic integrity
Each matrix cell score = (NSC priority) × (intervention approach score). 
Thus, a high cell score indicates an intervention that (1) addresses a statistically significant condition and (2) aligns with preservation ethics. 
The highest-scoring interventions across the matrix represent the recommended prioritization for resource allocation.
For example, Structural Instability (priority 9) × Mitigation-Shelter Coat (score 6) = 54, would suggest that this should be a top funding priority.


\begingroup
\footnotesize
\setlength\tabcolsep{3pt}
\begin{longtable}{>{\raggedright\arraybackslash}p{2cm} p{1.8cm} p{1.8cm} p{1.5cm} p{1.8cm} p{1.5cm} p{1.5cm}}
    \caption{Intervention matrix for adobe degradation.}
    \label{tab:intervention_matrix_adobe} \\ % Add caption here
    \toprule
    \textbf{NSC} & \textbf{Abstention} & \textbf{Mitigation} & \textbf{Recon} & \textbf{Substitution} & \textbf{Circum} & \textbf{Accel} \\
    \midrule
    \endfirsthead

    \multicolumn{7}{c}{\tablename\ \thetable\ -- \textit{Continued from previous page}} \\
    \toprule
    \textbf{NSC} & \textbf{Abstention} & \textbf{Mitigation} & \textbf{Recon} & \textbf{Substitution} & \textbf{Circum} & \textbf{Accel} \\
    \midrule
    \endhead

    \midrule
    \multicolumn{7}{r}{\textit{Continued on next page}} \\
    \endfoot

    \bottomrule
    \multicolumn{7}{p{\textwidth}}{\textit{Note: NSC = Necessary and Sufficient Condition, Recon = Reconstitution, Circum = Circumvention, Accel = Acceleration}} \\
    \endlastfoot
    General Mechanism (0) & Accept the condition of and the rate of continuing expansion (0) & & & & & Demolish damaged member or members, and do not replace (0) \\
    \midrule
    Structural Instability (Factor 3) (9) & & Apply protective shelter coat (54) & Install flashing (36) & Use fiber-reinforced adobe (18) & Adjust structural elements/bracing (45) & \\
    \midrule
    Sill Deterioration (Factor 1) (8) & & Apply lime-based plaster (48) & Rebuild sill with original mix (32) & Use durable materials (16) & & \\
    \midrule
    Lintel/Surface Issues (Factor 2) (7) & & Install drip edge (42) & Replace with identical materials (28) & Replace with Cedar (14) & Support adjustments (35) & \\
    \midrule
    Out of Plane (7) & & & & Rebuild wall with improved foundation (14) & Secondary bracing system (35) & \\
\end{longtable}
\normalsize
\endgroup

The adobe degradation intervention matrix (Table \ref{tab:intervention_matrix_adobe}) reflects the factor structure identified in the statistical analysis. Global Structural Instability (Factor 3) receives the highest priority score (9), consistent with its role as a latent variable encompassing multiple forms of damage. The highest-scoring intervention for this NSC is the application of a protective shelter coat (score: 54), which represents a mitigation approach that can slow water infiltration without altering the wall's structural configuration. Circumvention through structural bracing receives the second-highest score (45), reflecting the need to address the underlying instability while recognizing that bracing introduces non-original elements.

Sill Deterioration (Factor 1) is assigned a priority score of 8, reflecting its distinct nature as an independent failure mechanism. The recommended mitigation approach involves applying lime-based plaster (score: 48), which is compatible with the original adobe substrate and can be renewed periodically. Reconstitution through rebuilding with the original adobe mix receives a lower score (32), as it requires removal of deteriorated material and may result in loss of original fabric.

Surface and Lintel Interactions (Factor 2) are assigned a priority score of 7. The coupling between these two forms of damage, as revealed by the factor analysis, suggests that interventions must address both components simultaneously. The installation of drip edges represents a mitigation approach (score: 42) that can reduce water infiltration at the vulnerable lintel-wall interface. Circumvention through support adjustments (score: 35) may be necessary in cases where lintel deflection is contributing to coat cracking.

\begingroup
\footnotesize
\setlength\tabcolsep{3pt}
\begin{longtable}{>{\raggedright\arraybackslash}p{2cm} p{1.8cm} p{1.8cm} p{1.5cm} p{1.8cm} p{1.5cm} p{1.5cm}}
    \caption{Intervention matrix for foundation degradation.}
    \label{tab:intervention_matrix_foundation} \\ % Add caption here
    \toprule
    \textbf{NSC} & \textbf{Abstention} & \textbf{Mitigation} & \textbf{Recon} & \textbf{Substitution} & \textbf{Circum} & \textbf{Accel} \\
    \midrule
    \endfirsthead

    \multicolumn{7}{c}{\tablename\ \thetable\ -- \textit{Continued from previous page}} \\
    \toprule
    \textbf{NSC} & \textbf{Abstention} & \textbf{Mitigation} & \textbf{Recon} & \textbf{Substitution} & \textbf{Circum} & \textbf{Accel} \\
    \midrule
    \endhead

    \midrule
    \multicolumn{7}{r}{\textit{Continued on next page}} \\
    \endfoot

    \bottomrule
    \multicolumn{7}{p{\textwidth}}{\textit{Note: NSC = Necessary and Sufficient Condition, Recon = Reconstitution, Circum = Circumvention, Accel = Acceleration. Foundation interventions involving ground disturbance require archaeological survey and monitoring per Section 106 of the National Historic Preservation Act to prevent impacts to subsurface cultural resources.}} \\
    \endlastfoot

    General Mechanism (0) & Accept the condition of the section and rate of continuing expansion (0) & & & & & Demolish damaged member, and do not replace (0) \\
    \midrule
    Exposed Foundation Height (8) & & Add fill around the foundation (48) & & & Construct a barrier/retaining wall (40) & \\
    \midrule
    Foundation Stone Deterioration (7) & & Slope ground for drainage (42) & Replace stones with similar ones (28) & Replace with durable materials (14) & Bond exterior face to core (35) & \\
\end{longtable}
\normalsize
\endgroup

The foundation degradation intervention matrix (Table \ref{tab:intervention_matrix_foundation}) assigns the highest priority to Exposed Foundation Height (score: 8), consistent with the Elastic Net analysis. The critical role of moisture management, highlighted by the dominance of Cap Deterioration in our model, is further supported by pilot embedded monitoring at the site, which demonstrated rapid moisture infiltration through compromised surfaces \cite{FOUNReport2020}. The recommended mitigation approach involves adding fill around the foundation (score: 48) to reduce the exposed height and thereby minimize vulnerability to splash-back and direct precipitation impact. This intervention is reversible and does not alter the foundation itself, making it highly compatible with preservation standards. Circumvention through the construction of a barrier or retaining wall (score: 40) offers an alternative that may be appropriate in locations where adding fill would alter site drainage patterns or obscure archaeological features.
Foundation Stone Deterioration receives a priority score of 7. The recommended mitigation approach involves sloping the ground to improve drainage (score: 42), which addresses the underlying cause of deterioration rather than merely treating its symptoms. This approach is consistent with the principle of addressing NSCs rather than simply repairing damage. 

\section{Limitations and Future Work}

This study has several limitations that should be considered when interpreting results and applying the methodology to other contexts.
With n=67 wall sections and approximately 22 features, the feature-to-sample ratio approaches 1:3. This motivated the use of regularized regression to prevent overfitting, with Elastic Net selected based on benchmarking results (Appendix \ref{appendix:benchmarking}). 
While cross-validation mitigates this risk, the limited sample size reduces statistical power to detect subtle effects, particularly for treatment variables where small sample sizes for individual intervention types limit the ability to assess treatment efficacy. 
Future studies at larger adobe complexes could achieve more precise effect size estimates with increased statistical power.

This case study focuses exclusively on Fort Union National Monument. 
While the methodological framework is transferable, the specific findings (e.g., foundation height as primary predictor) may reflect FOUN-specific conditions such as local climate, soil properties, construction techniques, and maintenance history. 
Validation at additional Southwestern adobe sites (e.g., Tumacácori, Pecos National Historical Parks) is necessary to establish generalizability of the statistical relationships observed.
Our data represent a single temporal snapshot, precluding analysis of degradation rates, treatment effectiveness over time, or causal inference about intervention impacts. 
Several questions remain unanswered: (1) Are high-degradation walls deteriorating faster, or have they merely persisted longer in poor condition? (2) Did bracing successfully arrest movement, or were already-stable walls selectively braced? (3) Have treatments extended service life, even if degradation scores remain elevated?

Finally, it is important to acknowledge that the Total Degradation Score is a composite metric derived mathematically from the individual condition scores (e.g., Cap Deterioration, Height Loss). Consequently, the high $R^2$ values observed in the Elastic Net analysis partially reflect this mathematical dependency rather than a purely empirical discovery of unknown relationships. However, this analysis remains valuable as a form of sensitivity analysis: it empirically recovers the implicit weights of the scoring system, demonstrating which specific conditions (Cap Deterioration, Height Loss) are the dominant drivers of the overall vulnerability rating in practice. This confirms that the scoring system is functioning as intended, prioritizing walls with critical material loss and moisture issues.

Intervention matrices recommend specific materials (e.g., lime-based plaster) based on general preservation principles. Site-specific compatibility testing per ASTM D4404 or equivalent is required before implementation to ensure thermal expansion compatibility, salt migration prevention, and chemical stability with FOUN adobe mineralogy.

Longitudinal monitoring with repeated LiDAR scans and condition surveys (recommended interval: 5 years would enable survival analysis and causal inference methods.

% The intervention matrices represent prioritization recommendations based on statistical associations and preservation principles, but lack empirical validation. 
% We have not demonstrated athat implementing the highest-scoring interventions actually reduces degradation rates or extends structure longevity. 
% A controlled intervention study, treating matched wall sections with different approaches and monitoring outcomes, would provide this validation but requires multi-year commitment 


While we discuss slenderness, overturning moments, and bearing pressures qualitatively, finite element modeling could quantify stress distributions, factor-of-safety margins, and failure probabilities. 
Integration of geotechnical investigation (soil bearing capacity, settlement potential) with structural analysis would provide a quantitative basis for prioritizing sections at structural risk.

Lastly, the intervention matrices prioritize based on degradation severity and preservation standards but do not incorporate economic constraints. 
Real-world implementation requires cost estimates (labor, materials, equipment), budget limitations, and cost-effectiveness analysis (dollars per year of service-life extension). 
NPS facility managers need this economic dimension for practical decision-making.
Addressing these limitations represents a roadmap for advancing data-driven heritage preservation methodology.

\section{Conclusions}

This study applied correlation analysis, Factor Analysis, and Elastic Net regression to 67 wall sections at Fort Union National Monument to identify patterns of degradation and inform intervention prioritization. The multivariate analyses converged on three principal damage modes: Sill Deterioration (Factor 1), Surface and Lintel Interaction (Factor 2), and Global Structural Instability (Factor 3), together explaining 52.5\% of variance in the degradation data. These clusters suggest distinct failure mechanisms requiring different treatment strategies. Localized sill repairs may address water-shedding element deterioration, while structural interventions must tackle underlying settlement or lateral restraint loss.

Elastic Net analysis identified Height Loss and Cap Deterioration as the strongest predictors of degradation (standardized coefficients 43.0 and 54.5). A supplementary geometric validation ($N=38$) demonstrated that physical slenderness alone explains negligible variance ($r \approx 0$).
Tall walls warrant higher prioritization for preventive maintenance based on this intrinsic vulnerability signal, even when current damage is not yet visible.

Treatment variables (bracing, historical interventions) showed negligible importance, likely reflecting selection bias. Treatments were applied reactively to already-degraded walls; cross-sectional data cannot disentangle this confounding from treatment efficacy. Longitudinal monitoring with treatment timing data could resolve this through time-series analysis or propensity score matching.

The intervention matrices translate statistical priorities into preservation-compatible action frameworks by multiplicatively combining NSC importance (derived from factor variance and RF importance) with Secretary of Interior Standards compliance scores. Higher cell scores identify interventions addressing significant conditions through preservation-appropriate means (mitigation, circumvention). Field validation remains necessary to confirm these analytically-derived priorities match practical effectiveness.

Transferability to other adobe sites depends on several factors. The rapid assessment protocol and statistical workflow are generalizable; site-specific calibration would require adjustment for local climate, construction methods, and treatment history. Extensions incorporating temporal monitoring, climate scenario modeling, and cost data would strengthen decision-support utility. The limitations outlined in Section 6, particularly sample size, single-site scope, constrain extrapolation beyond FOUN without additional validation studies.

\section*{Acknowledgments}


\section*{Declaration of interest}

The authors report there are no competing interests to declare. 

\section*{Funding}
This material is based upon work supported by the National Science Foundation under Grant No. CMMI 2222849. Any opinions, findings, conclusions, or recommendations expressed in this material do not necessarily reflect the views of the National Science Foundation. 

\vspace{1cm}

\section*{Data availability statement}
The data that support the findings of this study, along with interactive educational notebooks demonstrating the methodology, are openly available in the GitHub repository at \url{https://github.com/rkn2/FOUN_ncptt}. The repository includes synthetic datasets generated to preserve the statistical properties of the original sensitive data, as well as Jupyter notebooks for interactive diagnostics and intervention planning.

\section*{Notes on contributor(s)}
\textbf{Conceptualization} (RN, MS, EOJ, FM, TB); \textbf{Methodology} (RN, MS, EOJ, FM, TB); \textbf{Writing - Original} Draft (RN, MS, EOJ, TB); \textbf{Writing - Review \& Editing} (RN, MS, EOJ, FM, TB).


% Start the appendix
\appendix

% First appendix section
\section{Data collection}
% Content of your first appendix
\subsection{RAS} \label{sec:ras}
CAN WE ADD THE RAS IN HERE?


\subsection{Bracing details} \label{sec:bracing}

\begingroup
\footnotesize
\setlength\tabcolsep{4pt}
\begin{longtable}{>{\raggedright\arraybackslash}p{2.5cm} >{\raggedright\arraybackslash}p{2.5cm} >{\raggedright\arraybackslash}p{3cm}}
    \caption{Structural bracing attribute \cite[]{mapbracing}}
    \label{tab:bracing} \\
    \toprule
    \textbf{Bracing Number} & \textbf{Wall ID} & \textbf{Material} \\
    \midrule
    \endfirsthead

    \multicolumn{3}{c}{\tablename\ \thetable\ -- \textit{Continued from previous page}} \\
    \toprule
    \textbf{Bracing Number} & \textbf{Wall ID} & \textbf{Material} \\
    \midrule
    \endhead

    \midrule
    \multicolumn{\textbf{3}}{r}{\textit{Continued on next page}} \\
    \endfoot

    \bottomrule
    \endlastfoot
    
    B108 & 5712A-W & 1'-1/2" Angle \\
    B109 & 5712B-W & 2" Tube \\
    B110 & 5712B-W & 2" Tube \\
    B111 & 5712C-W & 2" Wood \\
    B112 & 5712C-W & 2" Wood \\
    B113 & 5712C-W & 2" Tube \\
    B114 & 5717B-W & 1'-1/2" Angle \\
    B115 & 5717B-W & 1'-1/2" Angle \\
    B116 & 5721B-N & 1'-1/2" Angle \\
    B117 & 5725A-W & 2" Tube \\
    B118 & 5723A-N & 2" Tube \\
    B119 & 5729A-E & Wood Plate \\
    B120 & 5729B-E & Wood Plate \\
    B121 & 5729B-E & Wood Plate \\
    B122 & 5705A-W & Wood Plate \\
    B123 & 5705B-N & 1'-1/2" Angle \\
    B124 & 5705E-N & 1'-1/2" Angle \\
    B125 & 5705E-N & Wood Plate \\
    B126 & 5705E-S & Wood Plate \\
    B127 & 5705E-N & Wood Plate \\
    B128 & 5705E-S & Wood Plate \\
    B129 & 5706D-E & Wood Plate \\
    B130 & 5706D-W & Wood Plate \\
    B131 & 5728B-N & 1'-1/2" Angle \\
    B132 & 5724A-S & Wood Plate \\
    B133 & 5706A-W & Wood Plate \\
    B134 & 5718B-E & 1'-1/2" Angle \\
    B135 & 5715A-N & 1'-1/2" Angle \\
    B136 & 5715A-N & 1'-1/2" Angle \\
    B137 & 5715A-N & 1'-1/2" Angle \\
\end{longtable}
\normalsize
\endgroup


\section{Data Featurization} 

\subsection{Orientation Definitions} \label{sec:orient}
\begin{figure}[htbp]
    \centering
    \caption{Orientations in the dataset}
    \label{orientations}
\begin{tikzpicture}[font=\sffamily,>=Triangle]
\scriptsize
  % Draw the building layout
  \node[rectangle, draw, minimum width=6cm, minimum height=.4cm, line width=1.5pt] (building) {};
  
  % Define the orientation arrows
  \draw[->, line width=.6pt] ($(building.north)+(0,0.2)$) -- +(0,2cm) node[above] {North};
  \draw[->, line width=.6pt] ($(building.south)-(0,0.2)$) -- +(0,-2cm) node[below] {South};
  \draw[->, line width=1pt] ($(building.east)+(0.5,0)$) -- +(2.5cm,0) node[right] {East};
  \draw[->, line width=1pt] ($(building.west)-(0.5,0)$) -- +(-2.5cm,0) node[left] {West};
  % Add labels for the building orientations with rotation for Orientation 1
  \node[align=center, above=1.1cm of building.north, rotate=90] (o1) {Orientation 1\\ Orientation 4};
  \node[align=center, below=1.1cm of building.south, rotate=90] (o2) {Orientation 2\\Orientation 3};
  \node[align=center, left=.7cm of building.west] (o3) {Orientation 2\\Orientation 3};
  \node[align=center, right=.7cm of building.east] (o4) {Orientation 1\\Orientation 4};
  
  % Draw the building's elevation lines
  \draw[dashed, line width=0.5pt] (building.north) -- (building.south);
  \draw[dashed, line width=0.5pt] (building.east) -- (building.west);
  
  % Indicate the major elevations
  \node[align=center, fill=white, inner sep=2pt] at ($(building.center)!0.5!(building.north)$) {};
  \node[align=center, fill=white, inner sep=2pt] at ($(building.center)!0.5!(building.south)$) {};
  \node[align=center, fill=white, inner sep=2pt] at ($(building.center)!0.5!(building.west)$) {};
  \node[align=center, fill=white, inner sep=2pt] at ($(building.center)!0.5!(building.east)$) {};
  
\end{tikzpicture}
\end{figure}



\begingroup
\footnotesize
\setlength\tabcolsep{4pt}
\begin{longtable}{>{\raggedright\arraybackslash}p{3cm} p{9.5cm}}
    \caption{Metric definition in the dataset}
    \label{tab:metrics} \\ % Add caption here
    \toprule
    \textbf{Metric} & \textbf{Formula} \\
    \midrule
    \endfirsthead

    \multicolumn{2}{c}{\tablename\ \thetable\ -- \textit{Continued from previous page}} \\
    \toprule
    \textbf{Metric} & \textbf{Formula} \\
    \midrule
    \endhead

    \midrule
    \multicolumn{2}{r}{\textit{Continued on next page}} \\
    \endfoot

    \bottomrule
    \endlastfoot

    Coat Weight Score & Coat Loss Score + Coat Cracking Score \\
    \midrule
    Elevation Normalized Score & (Surface Loss at Low level + Surface Loss at Mid level + Surface Loss at Top level + Sill Damage Score) for each section in the specific orientation / Maximum Damage(Surface Loss at Low level + Surface Loss at Mid level + Surface Loss at Top level + Sill Damage Score) \\
    \midrule
    Elevation Weight Score & Elevation Norm Score $\times 1000$ \\
    \midrule
    Wall Normalize Score & (Cap Deterioration + Structural Cracking + Cracking at Wall Junction + Out of Plane + Lintel Deterioration + Height) for each section / Maximum Damage(Cap Deterioration + Structural Cracking + Cracking at Wall Junction + Out of Plane + Lintel Deterioration + Height) \\
    \midrule
    Wall Weight Score & Wall Normalize Score $\times 1000$ \\
    \midrule
    Total Score & Wall Weight Score + Coat1 Weight Score + Coat2 Weight Score + Coat3 Weight Score + Coat4 Weight Score + Elevation1 Weight Score + Elevation2 Weight Score \\
    \midrule
    Wall Rank & Wall Weight Score = Higher Wall Rank \\
\end{longtable}
\normalsize
\endgroup

\appendix

\section{Complete Model Benchmarking Results}
\label{appendix:benchmarking}

To ensure the generalizability of our model selection, we benchmarked seven regression model families using repeated K-fold cross-validation (5 folds $\times$ 5 repeats = 25 evaluation rounds). Table \ref{tab:model_benchmark_full} presents the complete results, including mean cross-validated $R^2$, standard deviations, and statistical comparisons using paired Wilcoxon signed-rank tests with Holm-Bonferroni correction.

\begin{table}[h]
\centering
\caption{Complete Model Benchmarking Results with Statistical Comparisons}
\label{tab:model_benchmark_full}
\begin{tabular}{lcccc}
\toprule
Model & Mean $R^2$ & Std Dev & $p_{raw}$ vs Best & $p_{adj}$ (Holm) \\
\midrule
Ridge & 0.764 & $\pm$0.132 & --- & --- \\
Elastic Net & 0.750 & $\pm$0.082 & 0.1730 & 0.1730 \\
Linear Regression & 0.744 & $\pm$0.144 & 0.0000 & 0.0000 \\
Gradient Boosting & 0.612 & $\pm$0.150 & 0.0001 & 0.0003 \\
Random Forest & 0.588 & $\pm$0.161 & 0.0004 & 0.0009 \\
Decision Tree & -0.007 & $\pm$0.537 & 0.0000 & 0.0000 \\
Naive Baseline (Mean) & -0.096 & $\pm$0.117 & 0.0000 & 0.0000 \\
SVR & -0.102 & $\pm$0.123 & 0.0000 & 0.0000 \\
\bottomrule
\multicolumn{5}{l}{\footnotesize Best model: Ridge ($R^2$ = 0.764)} \\
\multicolumn{5}{l}{\footnotesize Wilcoxon signed-rank test with Holm-Bonferroni correction ($\alpha=0.05$)} \\
\multicolumn{5}{l}{\footnotesize Repeated K-Fold Cross-Validation: 5 folds $\times$ 5 repeats = 25 evaluation rounds} \\
\end{tabular}
\end{table}

\begin{figure}[h]
    \centering
    \includegraphics[width=1.0\textwidth]{Images/calibration_plots.png}
    \caption{Calibration plots (Predicted vs. Observed) for all benchmarked models. Ideally, points should align with the red dashed line ($y=x$). Elastic Net and Ridge show strong alignment, while non-linear models like Random Forest show greater dispersion and bias.}
    \label{fig:calibration_plots}
\end{figure}

Ridge Regression achieved the highest mean cross-validated $R^2$ (0.764 $\pm$ 0.132), establishing it as the best-performing model. Elastic Net ($R^2$ = 0.750 $\pm$ 0.082) was statistically indistinguishable from Ridge based on paired Wilcoxon signed-rank testing with Holm-Bonferroni correction ($p_{adj} = 0.17 > 0.05$). 

All other models were significantly worse than Ridge:
\begin{itemize}
    \item Linear Regression: $p_{adj} < 0.001$
    \item Gradient Boosting: $p_{adj} = 0.0003$
    \item Random Forest: $p_{adj} = 0.0009$
    \item Decision Tree, SVR, and Naive Baseline: $p_{adj} < 0.001$ (negative $R^2$ indicates poor fit)
\end{itemize}

The statistical equivalence of Ridge and Elastic Net, combined with Elastic Net's superior interpretability through L1-induced sparsity, justified its selection as the primary model for feature importance analysis in the main manuscript.

\section{Educational Materials and Synthetic Data}
\label{appendix:educational}

To facilitate knowledge transfer and enable replication of this methodology at other heritage sites, we developed supplementary educational materials available in the project repository. These materials include interactive Jupyter notebooks and synthetic datasets designed to preserve data privacy while maintaining statistical fidelity.

\subsection{Interactive Jupyter Notebooks}

Two educational notebooks accompany this manuscript:

\textbf{1. Grant Methodology Demonstration} (\texttt{grant\_methodology\_demo.ipynb}): This notebook provides a step-by-step walkthrough of the complete analytical pipeline, including data preprocessing, correlation analysis, Factor Analysis, and Elastic Net regression. Each code cell is annotated with explanations of statistical assumptions, parameter choices, and interpretation guidelines. The notebook uses synthetic data (described below) to demonstrate the methodology without exposing sensitive site information.

\textbf{2. Intervention Matrix Development} (\texttt{intervention\_matrix\_notebook.ipynb}): This notebook illustrates the process of translating statistical findings into actionable intervention matrices. It demonstrates how Factor Analysis loadings and regression coefficients inform the identification of Necessary and Sufficient Conditions (NSCs), and how preservation standards are integrated into the priority scoring framework. The notebook includes worked examples of matrix construction and priority calculation.

\subsection{Synthetic Data Generation}

The actual Fort Union degradation data contains sensitive information about a National Historic Landmark and cannot be publicly shared without National Park Service approval. To enable educational use and methodological replication, we generated a synthetic dataset (\texttt{synthetic\_adobe\_data.csv}) that preserves key statistical properties of the real data while protecting site confidentiality.

The synthetic data generation process (\texttt{generate\_synthetic\_data.py}) implements the following design principles:

\begin{itemize}
    \item \textbf{Correlation preservation}: Key pairwise correlations (e.g., Height vs. Total Score, Cap Deterioration vs. Total Score) are maintained within $\pm$0.15 of observed values.
    \item \textbf{Factor structure}: The synthetic data exhibits three latent factors analogous to those identified in the real data (Sill Deterioration, Surface/Lintel Interaction, Structural Instability).
    \item \textbf{Distribution matching}: Marginal distributions for each variable approximate the real data distributions (e.g., Poisson for count-based damage scores, gamma for continuous geometric measurements).
    \item \textbf{Sample size consistency}: The synthetic dataset contains n=67 observations, matching the real Fort Union sample size to preserve feature-to-sample ratio constraints.
\end{itemize}

The synthetic data enables users to execute all analyses presented in this manuscript, verify computational results, and adapt the methodology to their own heritage sites without requiring access to restricted Fort Union data. Validation tests confirm that regression models trained on synthetic data yield qualitatively similar feature importance rankings and comparable cross-validated $R^2$ values (within 0.10 of real data results).

\subsection{Repository Access}

All educational materials, Python scripts, and documentation are available at the project repository: \url{https://github.com/rkn2/FOUN_ncptt}. The repository includes a comprehensive README with installation instructions, dependency requirements, and usage examples.

\section{Sensitivity Analysis of Sill Redundancy}
\label{appendix:sensitivity}

To address potential measurement redundancy between the highly correlated Sill 1 and Sill 2 variables ($r=0.99$), a sensitivity analysis was conducted. The two variables were consolidated into a single mean score (`Sill\_Mean`), and the Factor Analysis was re-run on the reduced set of 7 variables. As shown in Table \ref{tab:sensitivity_fa}, the resulting factor structure for Factors 2 and 3 remained substantively identical to the original model, confirming that the identification of ``Surface/Lintel Interaction'' and ``Structural Instability'' is robust to the handling of sill redundancy. Furthermore, the consolidated `Sill\_Mean` variable continued to load strongly on a distinct factor (loading 1.00), validating ``Sill Deterioration'' as a unique failure mode distinct from other degradation patterns.

\begin{table}[h]
\centering
\caption{Sensitivity Analysis: Factor Loadings with Consolidated Sill Variable}
\label{tab:sensitivity_fa}
\begin{tabular}{lccc}
\toprule
\textbf{Variable} & \textbf{Factor 1 (Sill)} & \textbf{Factor 2 (Surface/Lintel)} & \textbf{Factor 3 (Structural)} \\
\midrule
Sill\_Mean & \textbf{1.00} & 0.05 & -0.03 \\
Coat 1 Cracking & -0.11 & 0.43 & \textbf{0.60} \\
Coat 1 Loss & 0.09 & 0.14 & \textbf{0.51} \\
Lintel Deterioration & 0.12 & \textbf{0.82} & 0.10 \\
Coat 2 Cracking & 0.08 & \textbf{-0.54} & 0.03 \\
Structural Cracking & 0.07 & -0.33 & 0.36 \\
Out of Plane & -0.02 & -0.21 & \textbf{0.48} \\
\bottomrule
\end{tabular}
\end{table}

\section{Model Diagnostics}
\label{appendix:diagnostics}

Residual diagnostics for the Elastic Net model are presented in Figure \ref{fig:residual_diagnostics}. The Residuals vs Fitted plot (top left) shows a random scatter around zero, indicating constant variance (homoscedasticity) and linearity. The Normal Q-Q plot (top right) shows residuals following the theoretical normal line, confirming the normality assumption required for inference. The Scale-Location plot (bottom left) further supports homoscedasticity, and the Residuals vs Leverage plot (bottom right) identifies no points with high leverage and high residuals, suggesting the model is not unduly influenced by outliers.

\begin{figure}[h]
    \centering
    \includegraphics[width=0.9\textwidth]{Images/residual_diagnostics.png}
    \caption{Residual diagnostic plots for the Elastic Net model: (a) Residuals vs Fitted, (b) Normal Q-Q, (c) Scale-Location, and (d) Residuals vs Leverage.}
    \label{fig:residual_diagnostics}
\end{figure}

    \caption{Residual diagnostic plots for the Elastic Net model: (a) Residuals vs Fitted, (b) Normal Q-Q, (c) Scale-Location, and (d) Residuals vs Leverage.}
    \label{fig:residual_diagnostics}
\end{figure}

\section{Structural Stability Check Details}
\label{appendix:structural}

A simplified structural analysis was performed to distinguish between elastic instability (buckling) and tensile failure risks. The analysis used the following parameters and governing equations:

\textbf{Parameters:}
\begin{itemize}
    \item Modulus of Elasticity ($E$): 50,000 psi (conservative estimate for adobe).
    \item Wind Pressure ($q$): 20 psf.
    \item Effective Length Factor ($k$): 2.0 (Cantilever).
    \item Wall Thickness ($d$): Maximum profile width derived from LiDAR/DXF.
\end{itemize}

\textbf{Equations:}
\begin{itemize}
    \item Critical Buckling Load: $P_{cr} = \frac{\pi^2 E I}{(kL)^2}$
    \item Factor of Safety (Buckling): $FS = P_{cr} / P_{actual}$
    \item Net Tension: $\sigma_{net} = -\frac{P}{A} + \frac{Mc}{I}$
\end{itemize}

\textbf{Results:}
\begin{itemize}
    \item \textbf{Buckling:} Min FS = 1.19, Mean FS = 44.5. Zero walls failed ($FS < 1$).
    \item \textbf{Tension:} 95\% of walls developed net tension ($> 0$ psi).
\end{itemize}

\textbf{Conclusion:} Walls are geometrically stable against buckling but highly vulnerable to tensile cracking, validating the importance of condition-based predictors.

\bibliography{biblio.bib} 


\end{document}
